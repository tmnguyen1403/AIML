% !TEX root = //Users/amynguyen/Projects/MathAIML/latex_files/gradients.tex
\documentclass{article}
\usepackage{amsmath}
\begin{document}
The gradient
The gradient for f(x,y)

Understanding the gradient conceptutally  \\
\begin{itemize}
    \item The gradient represents the slope of a function.
    \item  The gradient points in the direction of the greatest rate of increase of the function,
    and its' magnitude is the slope in that direction.
\end{itemize}
\textbf{Gradient Vectors}\\
\begin{math}
    \nabla f(x,y,z) = [\frac{df(x,y,z)}{dx},\frac{df(x,y,z)}{dy},\frac{df(x,y,z)}{dz}]
\end{math} \\
\textbf{The Jacobian Matrix}\\
Gradient vectors organize the partial derivatives for a scalar function. If we have multiple functions, we use the Jacobian\\
\begin{math}
    \textbf{J}= 
    \begin{bmatrix}
        \nabla f(x,y)\\
        \nabla g(x,y)\\
    \end{bmatrix}
    = \begin{bmatrix}
        \frac{df(x,y)}{dx} & \frac{df(x,y)}{dy} \\
        \frac{dg(x,y)}{dx} & \frac{dg(x,y)}{dy} \\
    \end{bmatrix}
\end{math}\newline
\textbf{The general case}\\
\begin{math}
    f(x,y,z) \rightarrow f(x) where\  x = 
    \begin{bmatrix}
        x1\\x2\\.\\.\\.\\x_n
    \end{bmatrix}\\
    \textbf{J}= 
    \begin{bmatrix}
        \nabla f_1(x,y)\\
        \nabla f_2(x,y)\\
        ...\\
        \nabla f_n(x,y)\\
    \end{bmatrix}
    = \begin{bmatrix}
        \frac{df_1(\textbf{x})}{dx_1} & \frac{df_1(\textbf{x})}{dx_2} & ... & \frac{df_1(\textbf{x})}{dx_n} \\
        \frac{df_2(\textbf{x})}{dx_1} & \frac{df_2(\textbf{x})}{dx_2} & ... & \frac{df_2(\textbf{x})}{dx_n} \\
        ...\\
        \frac{df_m(\textbf{x})}{dx_1} & \frac{df_m(\textbf{x})}{dx_2} & ... & \frac{df_m(\textbf{x})}{dx_n} \\
    \end{bmatrix}
\end{math}
\end{document}